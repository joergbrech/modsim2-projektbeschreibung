\chapter{Equations}
For equations there are different ways to enter the mathematical environment. Using the command \verb+$ \frac{1}{x} $+ creates $ \frac{1}{x} $ within the text.

\begin{equation}\label{eq:linfnc}
F(x) = m \cdot x + b
\end{equation}

\begin{equation*}
F(x) = ax^{2}+bx+c
\end{equation*}
\noindent
Referencing Equation \ref{eq:linfnc}.\\
\newline
For matrices in brackets use \verb+$\begin{bmatrix} ... \end{bmatrix}$+ :
\begin{center}
$A=\begin{bmatrix}
1	& 0	& \dots	 & 0      \\
0	& 1 	& \dots  & 0 	  \\
\vdots	& 0 	& \ddots & \vdots \\
0 	& \dots & 0	 & 1
\end{bmatrix}$
\end{center}
\noindent
While for matrices in parentheses \verb+$\begin{pmatrix} ... \end{pmatrix}$+ is used :
\begin{center}
$B=\begin{pmatrix}
1	& 0	& \dots	 & 0      \\
0	& 1 	& \dots  & 0 	  \\
\vdots	& 0 	& \ddots & \vdots \\
0 	& \dots & 0	 & 1
\end{pmatrix}$
\end{center}
\noindent 
Using just \verb+$\begin{matrix} ... \end{matrix}$+ yields a matrix without brackets or parentheses:
\begin{center}
$\begin{matrix}
1	& 0	& \dots	 & 0      \\
0	& 1 	& \dots  & 0 	  \\
\vdots	& 0 	& \ddots & \vdots \\
0 	& \dots & 0	 & 1
\end{matrix}$
\end{center}